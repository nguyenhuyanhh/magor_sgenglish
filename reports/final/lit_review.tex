\chapter{Literature Review}

This chapter reviews the currently available implementations of various
components in the processing pipeline. The first two sections enumerate
the two relevant pipelines --- Section 2.1 describes transcription, and
Section 2.2 describes captioning. Finally, Section 2.3 summarises the
various implementations.

Most of the implementations chosen are free, open-source and cross-platform.

\section{Transcription}

Figure~\ref{trans} provided the transcription pipeline of three components
--- a resampler, a diarizer and a transcription engine. We would go into
each component in detail.

Furthermore, to enable transcription of multi-channel recording, the concept
of voice-activity detection (VAD) would also be discussed.

\subsection{Resampling}

In the context of this project, resampling is concerned with converting an
audio stream from different specifications to a standard one, before passing to
other processing steps. Some common attributes of an audio stream are:

\begin{longtabu}{X[1.5,l]X[4,l]X[2.5,l]}
    \textbf{Attribute} & \textbf{Description} & \textbf{Examples} \\
    \midrule
    \endhead{}
    Sample rate &
    (defined for PCM audio)\newline Number of audio samples per
    second~\cite{weik1995communications} &
    CD audio: 44100 Hz~\cite{cd} \\
    Bit depth &
    (defined for PCM audio)\newline Number of bits to represent
    a sample~\cite{thompson2005understanding} &
    CD audio: 16 bits~\cite{cd} \\
    Number of channels &
    Number of independent audio channels (to create a
    perception of depth) &
    Mono (1-channel)/\newline
    Stereo (2-channel)~\cite{mono-stereo} \\
    Audio coding format &
    The specific encoder/ decoder used to create the audio stream;
    usually associated with a certain file extension &
    MPEG-2 Audio Layer III (\texttt{.mp3})~\cite{mp3}\newline
    WAVE (\texttt{.wav})~\cite{wav} \\
    \caption{Common audio stream attributes}
\end{longtabu}

Respectively, in order to resample audio streams, the following tasks
are performed on the original audio stream\footnote{The canonical
definition of resampling is only concerned with the first task
(sample rate conversion).}:

\begin{itemize}
    \item Sample rate conversion --- changing the sample rate of the audio
    (for instance, from 44100Hz to 16000Hz)
    \item Sample format conversion --- changing the type of the sample
    (for instance, from 16-bit to 8-bit samples)
    \item Channel rematrixing --- changing the number of channels
    (for instance, from stereo to mono audio)
    \item Transcoding --- changing the audio coding format (for instance,
    from WAVE to MPEG Layer-III)
\end{itemize}

There are two software packages to perform all the above tasks:

\subsubsection{Sound eXchange --- SoX (\texttt{sox})}

SoX is a cross-platform command-line utility that supports conversion
between a wide range of audio formats. Additionally, the utility could
apply effects, and play and record audio files~\cite{sox-docs}.
SoX is written in C\@; the associated library is \texttt{libsox}. Its
resampling library is released separately as
\texttt{libsoxr}~\cite{sox}.

Figure~\ref{sox} shows an example run of SoX.

\begin{figure}[ht]
\begin{center}
    \includegraphics[width=0.6\textwidth]{sox}
    \caption{An example SoX session~\cite{sox}}\label{sox}
\end{center}
\end{figure}

The latest version of SoX, 14.4.2 was released in February 2015. The
current status of development work is uncertain; many issues and pull
requests are outstanding~\cite{sox-cl}.

\subsubsection{FFmpeg (\texttt{ffmpeg})}

FFmpeg is another C cross-platform utility to record, convert and stream
audio and video~\cite{ffmpeg}. FFmpeg provides a resampler which can
perform audio resampling, audio channel layout rematrixing, and convert
audio format and packing layout; it interfaces with its own resampling
library (\texttt{libswresample}) as well as SoX's resampling
library~\cite{ffmpeg-res,ffmpeg-libres}.

The library is actively developed; its latest stable version is 3.4
and it was released in October 2017~\cite{ffmpeg-dl}.

\subsubsection{Evaluation}

The two packages offer very similar functionalities; they are also
highly performant. However, FFmpeg has two clear advantages: it is
actively maintained and updated, and it operates on both audio and
video files (compare to audio only for SoX). FFmpeg also supports
more audio formats out-of-the-box.

Overall, FFmpeg would be a more suitable package for the project.

\subsection{Diarization}

With respect to an audio stream of multiple speakers, diarization is
the process of determining which speaker is speaking at when~\cite{diar}. 
Diarization is a combination of two tasks --- speaker segmentation,
which is the process of finding speaker change points in an audio stream,
to split the stream into smaller segments; and speaker clustering,
which is the process of grouping said segments based on speaker
characteristics, ideally resulting in one cluster per actual
speaker~\cite{diar-cls}.

Implementations of speaker diarization usually follow one of these
two approaches~\cite{diar}:

\begin{itemize}
    \item Bottom-up --- initialize a large number of segments (more than
    the optimal), then iteratively merge segments until optimal
    \item Top-down --- initialize a single model of one segment (the whole
    audio stream), then iteratively add new models corresponding to new
    segments, until optimal
\end{itemize}

Both are characterised by the use of Hidden Markov Models (HMMs) and 
Gaussian Mixture Models (GMMs), a statiscal and probabilistic approach
to clustering.

One of the most frequently cited diarization toolkit is LIUM Speaker
Diarization.

\subsubsection{LIUM Speaker Diarization}

LIUM Speaker Diarization is an open-source diarization toolkit developed
at Laboratoire d'Informatique de l'Université du Maine~\cite{lium}.

\begin{figure}[h]
\begin{center}
    \includegraphics[width=0.6\textwidth]{diarization}
    \caption{The LIUM processing pipeline~\cite{lium-ovr}}\label{lium}
\end{center}
\end{figure}

Figure~\ref{lium} outlines the steps in LIUM's processing pipeline.
LIUM follows the bottom-up approach; it utilises agglomerative hierarchical
clustering, using Bayesian Information Criterion as the stopping/ clustering
criteria. LIUM produces a segmentation file (\texttt{.seg}) which details
the start time, length and \texttt{speaker\_id} of all identified
segments. An example segmentation is shown in Figure~\ref{seg}\footnote{The
segmentation output has 8 fields, in this particular
order: \texttt{show\_id}, \texttt{channel\_id}, start time, length, gender
type of band, type of environment and \texttt{speaker\_id}.}~\cite{lium-seg}.

\begin{figure}[h]
\begin{lstlisting}
20071218_1900_1920_inter 1 0    322   M S U S0
20071218_1900_1920_inter 1 322  680   F S U S1
20071218_1900_1920_inter 1 1148 371   M S U S143
20071218_1900_1920_inter 1 1772 310   F S U S3
20071218_1900_1920_inter 1 2082 318   F S U S28
20071218_1900_1920_inter 1 2495 1570  M S U S12
\end{lstlisting}
\caption{Sample segmentation output}\label{seg}
\end{figure}

LIUM is written in Java; however there is a convenient CLI~\cite{lium}
utilising its \texttt{.jar} package.
Its latest version is 8.4.1, released in September 2013~\cite{lium-dl}.

\subsubsection{Evaluation}

LIUM is considered one of the state-of-the-art toolkits in speaker
diarization; it came out on top in a series of evaluations~\cite{lium-ester,
lium-revere}. It is also very performant, as evidence in Figure~\ref{lium}:
using relatively recent hardware, it takes less than half-real-time
to diarize an audio stream. It also does a good job of preparing the
audio stream for ASR\@: resulting segments are short and single-speaker.

However, the out-of-the-box configuration is tuned towards broadcast
news diarization; for other audio domains the results might vary
and specific tuning might be required. The toolkit is also fairly outdated.

Nonetheless, LIUM is a great starting point for building the diarization
component for the transcription processing pipeline. 

\subsection{Transcription Engine}

The main purpose of the transcription engine is to transcribe speech
segments to text (in other words, to perform ASR). 

Figure~\ref{asr} outlines the typical architecture of an ASR
system. The ASR system takes in an audio stream and extract
its features --- the most common being mel-frequency cepstral
coeffcients (MFCCs) --- then pass the features to an acoustic model.
Combined with a language model, the ASR system comes up with probable
results and selects the best one based on its criteria~\cite{asr}.

\begin{figure}[h]
\begin{center}
    \includegraphics[width=0.9\textwidth]{pipeline_asr}
    \caption{Structure of an ASR system}\label{asr}
\end{center}
\end{figure}

The current state-of-the-art in ASR are among these two methodologies:

\begin{itemize}
    \item Statiscal/ probabilistic methods, utilising a combination of
    HMMs and GMMs
    \item Deep neural network (DNN)-based methods
\end{itemize}

With the advent of high-performance computing, DNN-based methods are
quickly gaining traction. We review two DNN-based transcription engines.

\subsubsection{Google Cloud Speech API}

Cloud Speech API is developed by Google, providing the same speech
recognition capabilities powering Google's native applications (such as
Google Voice Search and Google Assistant) to developers. Cloud Speech API
has support for 110 languages, including multiple variances of popular
languages such as English~\cite{gcs}.

The implementation details of Cloud Speech API are hidden to the developer;
various API libraries\footnote{Most of the libraries are experimental and
continuously updated.} for common programming languages are
provided~\cite{gcs-libs}. In a normal workflow, speech segments are sent to
Google's servers through the client libraries in one of three configurations:
synchronous, asynchronous and streaming recognition; the servers return
recognition results in JSON\@. A sample is given in Figure~\ref{gcs}.

\begin{figure}[ht]
\begin{lstlisting}
    {
        "results": [
            {
                "alternatives": [
                    {
                        "transcript": "how old is the Brooklyn Bridge",
                        "confidence": 0.98267895
                    }
                ]
            }
        ]
    }      
\end{lstlisting}
\caption{Sample Google Cloud Speech API recognition result}\label{gcs}
\end{figure}

To perform speech recognition, one needs to provide
a Google API service account key~\cite{gcs-api-key} and this key is used
for usage tracking; the free tier only allows \$300 of credits to be used
in 12 months~\cite{gcs-free}\footnote{Usage is charged in terms of 15-second
blocks; each block is \$0.0025.}.

\subsubsection{Kaldi}

In SLRG, there have been efforts to build a LVCSR system for the English
language~\cite{slrg}. This effort is led by Dr.\ Li Haihua; the system is based
on \texttt{Kaldi}~\cite{kaldi}, an open-source toolkit for speech
recognition written in C++. The system is provided as a black box; the models
and executable are conveniently packaged together. A Python wrapper has been
implemented.

\subsubsection{Evaluation}

\subsection{Voice-activity Detection (VAD)}

In a multi-channel recording scenario.

\section{Captioning}